\documentclass[a4paper,11pt]{article}

\usepackage[T2A]{fontenc}
\usepackage[utf8]{inputenc}
\usepackage[russian]{babel}

% %A Few Useful Packages
\usepackage{marvosym}
%\usepackage{fontspec} 					%for loading fonts
\usepackage{url,parskip} 	%other packages for formatting
\RequirePackage{color,graphicx}
\usepackage[usenames,dvipsnames]{xcolor}
\usepackage[big]{layaureo} 				%better formatting of the A4 page
% an alternative to Layaureo can be ** \usepackage{fullpage} **
\usepackage{supertabular} 				%for Grades
\usepackage{titlesec}					%custom \section
\usepackage{enumitem}
\usepackage{xspace}



\usepackage{hyphenat}


%Setup hyperref package, and colours for links
\usepackage{hyperref}
\definecolor{linkcolour}{rgb}{0,0.2,0.6}
\hypersetup{colorlinks,breaklinks,urlcolor=linkcolour, linkcolor=linkcolour}

%FONTS
%\defaultfontfeatures{Mapping=tex-text}
%\setmainfont[SmallCapsFont = Fontin SmallCaps]{Fontin}
%%% modified for Karol Kozioł for ShareLaTeX use
% \setmainfont[
% SmallCapsFont = Fontin-SmallCaps.otf,
% BoldFont = Fontin-Bold.otf,
% ItalicFont = Fontin-Italic.otf
% ]
% {Fontin.otf}
%%%

%CV Sections inspired by: 
%http://stefano.italians.nl/archives/26
\titleformat{\section}{\Large\scshape\raggedright}{}{0em}{}[\titlerule]
\titlespacing{\section}{0pt}{0pt}{0pt}
%Tweak a bit the top margin
%\addtolength{\voffset}{-1.3cm}

\newcommand{\CC}{C\nolinebreak\hspace{-.05em}\raisebox{.4ex}{\tiny\bf +}\nolinebreak\hspace{-.10em}\raisebox{.4ex}{\tiny\bf +}}

% %Italian hyphenation for the word: ''corporations''
% %\hyphenation{im-pre-se}

%\addtolength{\textwidth}{1.0in}

% %-------------WATERMARK TEST [**not part of a CV**]---------------
\usepackage[absolute]{textpos}

\setlength{\TPHorizModule}{30mm}
\setlength{\TPVertModule}{\TPHorizModule}
\textblockorigin{2mm}{0.65\paperheight}
\setlength{\parindent}{0pt}

%--------------------BEGIN DOCUMENT----------------------
\begin{document}
\newgeometry{left=0.5in,right=0.5in,top=0.5in,bottom=0.1in}

%WATERMARK TEST [**not part of a CV**]---------------
%\font\wm=''Baskerville:color=787878'' at 8pt
%\font\wmweb=''Baskerville:color=FF1493'' at 8pt
%{\wm 
%	\begin{textblock}{1}(0,0)
%		\rotatebox{-90}{\parbox{500mm}{
%			Typeset by Alessandro Plasmati with \XeTeX\  \today\ for 
%			{\wmweb \href{http://www.aleplasmati.comuv.com}{aleplasmati.comuv.com}}
%		}
%	}
%	\end{textblock}
%}

\pagestyle{empty} % non-numbered pages


%--------------------TITLE-------------
\par{\centering
		{\Huge Ичалов Леонид Евгеньевич
	}\bigskip\par}

%--------------------SECTIONS-----------------------------------
%Section: Personal Data
\begin{center}
%\section{Персональные данные}
%\hline
\hspace{0.7cm}\begin{tabular}{rl}
    \textsc{Дата рождения:} &  17 августа 2001 \\
    %\textsc{Адрес:}   & Москва, ул. Люсиновская, 68--446А \\
    \textsc{Телефон:}     & +7 915 242-00-57\\
    \textsc{email:}     & \href{mailto:leonid.ichalov@gmail.com}{leonid.ichalov@gmail.com}
\end{tabular}
\end{center}
\section{Образование}

\vspace{5px}
\begin{tabular}{lr}
\textbf{\large Высшая школа экономики}\\
Факультет компьютерных наук, образовательная программа
прикладная математика\\\hspace{14.35cm} и информатика \\
\hspace{0.5cm}Бакалавриат, 2 курс, 2024\\

\textbf{ГБОУ Школа № 179}, 2019\\
\end{tabular}
\\
\section{Дополнительное образование}

\vspace{5px}
\begin{tabular}{lr}
Тинькофф Поколение: машинное обучение, глубокое обучение\\
Летняя компьютерная школа, параллель A
\end{tabular}
\\

\section{Навыки}

\vspace{5px}
\begin{tabular}{lr}
Python, \CC, Linux, Jupyter\\
\end{tabular}
\\
\section{Другое}

\vspace{5px}
\begin{tabular}{l}
Победитель (2019) и призер (2017, 2018) заключительного этапа Всероссийской \\\hspace{3.35cm}олимпиады школьников по информатике
\end{tabular}

\end{document}


% 1. Образование пишется с текущего до самого старого. То есть сначала вышка, потом школа, все с годами обучения или предполагаемыми датами
% 2. Тиньков в дополнительное образование, туда можно дописать про ЛКШ, например (или еще что-нибудь). в тинькове тоже можно указать год и что не только МО но и нейросети (лучше полные названия курсов) 
% 3. Давай ориентироваться хотя бы на полстраницы. сейчас у тебя явно что-то с вертикальными отступами и в целом между каждой строчкой можно побольше пробелов и побольше шрифт =) И жирным места обучения
% 4. Теперь про достижения. Я сходу могу назвать что ты не только победитель но и призер, можно так: Победитель (2019) и призер (2017, 2018) ... 
% Потом ты вроде бы много был ассистентом. Был призером каких-то мат олимпиад 
% А, мб можно про юниверсум сказать? мб даже код есть?
% и ВКОШП!

% я пошла спать (точнее работать), пришли потом еще раз =)

% Что за влкдака строчка? тебе не обязательно все делать как у меня, сделай как считаешь красивым

% между словом и открывающей скобкой ставится пробел, это даже в программировании так! 
даже перед параметрами функции?
% на все у тебя есть ответ!)
Clion за меня все сделает. Меня достаточно нажать 3 клавиши(не помню какие, но клавиши)
Ctrl+Alt+L
% пиши в блокноте то есть виме
У меня произошла классная история с вимом
Я у друга спросил, кто его подсадил на вим
И он ответил, что я
% =D

там в дополнительном только тиньков и будет
только если ЛКШ
Меня просто бесит конструкция "вкладка-строчка"
Там прекрасное название "тинькофф образование, поколение: машинное обучение, глубокое обучение"
Я на мат. олимпиады не хожу с 9 класса % а зря :) даже на регион?
ну их меня выбесила их субъективность % это ты еще на олимпиадах по литературе не был
был, на школьном этапе